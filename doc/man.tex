%% $RCSfile$
%% $Revison$
%% $Date$
%% $Auther: XiangDong Tan (c) Copyright 1999

\documentstyle[11pt,psfig,spacing,subfigure,mysec]{article}
\renewcommand{\thefootnote}{\fnsymbol{footnote}}
\newcommand{\dummy}[0]{\rule{0mm}{0mm}}
%%\newfont{\caphelv}{rphvr scaled \magstep0}
\newcommand{\mycaption}[1]{\caption{\small #1}}
\newcommand{\proof}[1]{{\it Proof.\/} #1}

\input macros
\newcommand{\Xvk}{{X_{v_{k}}}}
\newcommand{\Xvkb}{{X_{\overline{v}_{k}}}}
\def\ba{\mbox{\bf a}}
\def\bb{\mbox{\bf b}}
\def\bc{\mbox{\bf c}}
\def\bx{\mbox{\bf x}}
\def\by{\mbox{\bf y}}
\def\bd{\mbox{\bf d}}
\def\be{\mbox{\bf e}}
\def\br{\mbox{\bf r}}
\def\bC{\mbox{\bf C}}
\def\bA{\mbox{\bf A}}
\def\bG{\mbox{\bf G}}
\def\bC{\mbox{\bf C}}
\def\bB{\mbox{\bf B}}
\def\bT{\mbox{\bf T}}
\def\bI{\mbox{\bf I}}
\def\bM{\mbox{\bf M}}
\def\bP{\mbox{\bf P}}
\def\bY{\mbox{\bf Y}}
\def\bu{\mbox{\bf u}}
\def\bs{\mbox{\bf s}}
\def\bt{\mbox{\bf t}}
\def\br{\mbox{\bf r}}
\def\bgamma{\mbox{\boldmath $\gamma$}}
\def\by{\mbox{\bf y}}
\def\bw{\mbox{\bf w}}
\def\union{\cup}
\def\intersect{\cap}
\def\bif{\mbox{\bf if }}
\def\bfor{\mbox{\bf for }}
\def\bdo{\mbox{\bf do }}
\def\breturn{\mbox{\bf return }}

%%
%% Picture macro (assuming postscript printer)
%% Usage: \PSfig{filename}{caption}{label}{height}{width}
%%
\newcommand{\psfigure}[6]
	 {
         \begin{figure}[#6]
	 \begin{center}
	 \ \psfig{figure=#1.ps,height=#4,width=#5}
         \vspace{-0.1in}
         \mycaption{#2}
         \label{#3}
	 \end{center}
         \vspace{-0.25in}
         \end{figure}
         }
 
\newcounter{numberlistc}
\newenvironment{numberlist}
    {   \setcounter{numberlistc}{0}
        \begin{list}{\arabic{numberlistc}.}
        {\usecounter{numberlistc}
        \setlength{\parsep}{0pt}
        \setlength{\topsep}{3pt}
        \setlength{\itemsep}{0pt}}
        }{ \end{list} }
\newcounter{itemlistc}
\newcounter{enumlistc}
%\renewcommand{\theromanlistc}{(\roman{romanlistc})} %for ref use
%\renewcommand{\thealphlistc}{(\alph{alphlistc})}    %for ref use
\newenvironment{itemlist}
    {   \setcounter{itemlistc}{0}
	\begin{list}{$\bullet$}
        {\usecounter{itemlistc}
        \setlength{\parsep}{0pt}
        \setlength{\topsep}{3pt}
        \setlength{\itemsep}{0pt}}
        }{ \end{list} }

%set dimensions of columns, gap between columns, and paragraph indent 
\setlength{\textheight}{8.75in}
\setlength{\columnsep}{2.0pc}
\setlength{\textwidth}{6.8in}
\setlength{\footheight}{0.0in}
\setlength{\topmargin}{0.25in}
\setlength{\headheight}{0.0in}
\setlength{\headsep}{0.0in}
\setlength{\oddsidemargin}{-.19in}
\setlength{\parindent}{1pc}

\begin{document}

\title{\Large\bf
\vspace{1.5in}
SCAD3 User Guide ($Revision$)\footnote{This work
is sponsored by U.S. Defense Advanced Research Projects Agency
(DARPA)
under grant number F33615-96-1-5601 from USAF,
Wright Laboratory, Manufacturing Technology Directorate,
and by Rockwell Semiconductor Systems.
}}

\vspace{4.0in}
\author{
\begin{tabular}[t]{c@{\extracolsep{1em}}c}
\multicolumn{2}{c}{by}\\
\multicolumn{2}{c}{Xiang-Dong Tan \hspace{0.3in} C.-J. Richard Shi}\\
\\
\multicolumn{2}{c}{Department of Electrical Engineering} \\
\multicolumn{2}{c}{University of Washington, Seattle, WA 98195} \\
\end{tabular}}
\date{May 5, 1999}

\vspace{3in}
\maketitle
\section*{\centering \Large Abstract}
\noindent
We desribe the program SCAD3, a symbolic circuit analyzer. 
SCAD3 stands for Symbolic Circuit Analysis with Determinant
Decision Diagrams. Informaiton is given about
the introduction of the program, the underlying
algorithms, along with installing, invoking and using the program. 
Current version is $Revision$.
\vspace{2.0pc}
\\
{\bf Key words:} Symbolic analysis, circuit simulation, BDD, CAD.

\newpage
\section{Introduction}
SCAD3 is a symbolic circuit analyzer developed at the
Unversity of Iowa and the University of Washington. 
It is based on a new approach to
exact and canonical symbolic analysis of
large analog circuits~\cite{ShiTan97}.
The new method consists of representing the
symbolic determinant of a circuit matrix
by a graph---called determinant decision diagram (DDD)---and 
performing symbolic analysis by graph manipulations.
The program is written in C++ programming lanaguage.
It accepts netlists in SPICE format. The nonlinear analog
circuits have to linearized at steady operational points
before it can be analyzed by SCAD3.

Key features of SCAD3 include
\begin{itemize}
\item Exactly analyze large analog circuits with circuit size 
up to 30 nodes.
\item Perform the hierarichal symbolic circuit analysis. 
\item Derive the exact s-expanded transfer function of analog circuit.
\item Perform approximation on small-signal characteristics such as 
transfer functions. 
\item Extract poles and zeros under pole spliting conditons.
\item Perform circuit-noise analysis and modeling.
\end{itemize}

\section{Symbolic Analysis Background}
Symbolic analysis is to calculate the behavior or
the characteristic of a circuit in terms of symbolic parameters.
It offers many advantages than numerical simulation
in many applications such as optimum topology selection, 
design space exploration, behavioral model generation, 
and fault detection~\cite{GiWa94}.
However, symbolic analysis has not
been widely used by analog designers,
and was once judged as completely inefficient~\cite{McPe71}.
The root of the difficulty is that:
the number of product terms in a symbolic expression
may increase exponentially with the size of a circuit.
For example, for a BiCMOS amplifier that has
about 15 nodes and 25 devices (transistors, diodes,
resistors and capacitors), the determinant
of the circuit matrix contains more 
than $10^{11}$ product terms~\cite{WaGi96}.
Any manipulation and evaluation
of symbolic expressions will require CPU time
at best linear in the number of terms, 
and therefore have both the time and space complexities
exponential in the size of a circuit.

To cope with large analog circuits,
modern symbolic analyzers~\cite{FeRo96}
rely on two techniques---hierarchical decomposition
and symbolic simplification---developed in the past decade.
Hierarchical decomposition is to generate
symbolic expressions in the nested instead of
the expanded form~\cite{HaLi95,StKo86}.
Symbolic simplification is to discard those insignificant
terms based on the relative magnitudes of 
symbolic parameters and the frequency defined at some
nominal design points or over some ranges.
It can be performed before/during the generation of
symbolic terms~\cite{ChMa92,HsSe94,SeDe92,YuSe96} 
or after the generation~\cite{FeMa92,GiSa91,WaGi96}.
Exploitation of these techniques has enabled the use
of symbolic simulators in several university research 
projects~\cite{CaCa91,FeRo96,GiSa91,YoSa95}.
However, both techniques have some major deficiencies.
Manipulation (other than evaluation) of a nested expression
usually requires complicated and time-consuming
procedures; e.g., 
sensitivity calculation in~\cite{Li92} and
lazy approximation in~\cite{SeDe92}.
On the other hand, simplified expressions only have a sufficient accuracy
at some points or frequency ranges.
Even worse, simplification often losses certain information,
such as sensitivity with respect to parasitics, 
which is crucial for circuit optimization and testability analysis.

SCAD3 is based  on a new approach to
exact symbolic analysis, which is capable of analyzing
analog integrated circuits substantially 
larger than those previously handled.
It is based on the two observations on
symbolic analysis of large analog circuits:
(a) the circuit matrix is sparse
and (b) a symbolic expression often shares many sub-expressions.
Under the assumption that all the matrix elements are distinct,
each product term involves a subset of all the
symbolic parameters. Therefore, we adapt 
Zero-suppressed Binary Decision Diagrams (ZBDDs)---introduced 
originally for representing sparse subset systems~\cite{Mi93}---to 
represent a symbolic determinant.
This leads to a new interpreted graph, 
called Determinant Decision Diagram (DDD).
The DDD representation has several advantages over both the
expanded and arbitrarily nested forms of a symbolic expression.
First, similar to the nested form,
the DDD representation is 
compact for a large class of analog circuits.
A ladder-structured network can be represented
by graphs where the number of vertices (called the DDD size)
grows linearly in the number of symbolic parameters.
Second, similar to the expanded form,
our representation is canonical, i.e.,
every determinant has a unique representation.
Finally, evaluation and manipulation of
symbolic determinants (such as sensitivity calculation and
approximation)
have time complexity proportional to the DDD size.

\section{Data Preparation}


\section{Usage Guide}
\label{sec:usage}

\begin{thebibliography}{99}

\bibitem{Br86}
R.~E. Bryant,
``Graph-based algorithms for Boolean function manipulation",
{\it IEEE Trans. Computer}, pp.~677--691, 1986.

\bibitem{CaCa91}
R. Carmassi, M. Catelani, G. Iuculano, A. Liberatore,
S. Manetti, and Marini,
``Analog network testability measurement:
a symbolic formulation approach",
{\it IEEE Trans. Instrumentation and Measurement},
vol.~40, pp.~930--935, Dec. 1991.

\bibitem{ChMa92}
S.-M. Chang, J.-F. MacKey, and G.~M. Wierzba,
``Matrix reduction and numerical approximation
during computation techniques for symbolic analog
circuit analysis",
pp.~1153--1156 in
{\it Proc. IEEE Int. Symp. Circuits and Systems},
1992.

\bibitem{maple5}
B.~W. Char, {\it et. al.},
{\it Maple V: Language Reference Manual},
Springer-Verlag,  1991.

\bibitem{FeMa92}
F.~V. Fern\'{a}ndez, J.D.~Mart\'{\i}n, A. Rodr\'{\i}guez-V\'{a}zquez 
and J.~L. Huertas, 
``On simplification techniques for symbolic
analysis of analog integrated circuits", 
pp.~1149--1152 in {\it Proc. IEEE Int. Symp. Circuits and Systems}, 1992.

\bibitem{FeRo96}
F.~V.~Fern\'andez and A.~Rodr\'{\i}guez-V\'azquez,
``Symbolic analysis tools---the state of the art", pp.~798--801 in 
{\it Proc. IEEE Int. Symposium on Circuits and System}, 1996.

\bibitem{GiSa91}
G. Gielen and W. Sansen,
{\em Symbolic Analysis for Automated Design of Analog Integrated Circuits}, 
Kluwer Academic Publishers, 1991.

\bibitem{GiWa94}
G.~Gielen, P.~Wambacq and W.~Sansen,
``Symbolic analysis methods and applications for 
analog circuits: A tutorial overview",
{\it Proc. IEEE}, vol.~82, no.~2, pp.~287--304, Feb. 1994. 

\bibitem{HaLi95}
M.~M.~Hassoun and P.~M.~Lin,
``A hierarchical network approach to symbolic analysis of large
scale networks",
{\it IEEE Trans. Circuits and Systems}, vol.~42, no.~4,
pp.~201--211, April 1995.

\bibitem{HsSe94}
Jer-Jaw Hsu and Carl Sechen, 
``DC small signal symbolic analysis of
large analog integrated circuits",
{\it IEEE Trans. Circuits and Systems}, vol.~41, no.~12,
pp.~817--828, Dec. 1994.

\bibitem{Li91}
P.~M. Lin, 
{\em Symbolic Network Analysis}, Elsevier Science Publishers B.V., 
1991.

\bibitem{Li92}
P.~M. Lin, ``Sensitivity analysis of large linear networks 
using symbolic program",
pp.~1145--1148 in {\it Proc. IEEE Int. Symp. Circuits and Systems}, 1992.

\bibitem{McPe71}
W.~J. McCalla and D.~O. Pederson,
``Elements of computer-aided analysis",
{\it IEEE Trans. Circuit Theory},
vol.~18, pp.~14--26, 1971.

\bibitem{Mi93}
S. Minato, 
``Zero-suppressed BDDs for set manipulation in combinatorial 
problems", pp. 272--277 in 
{\it Proc. 30th IEEE/ACM Design Automation Conference}, Dallas, TX, June 1993.

\bibitem{SeDe92}
S.~J.~Seda, M.~G.~R.~Degrauwe and W.~Fichtner,
``Lazy-expansion symbolic expression approximation in SYNAP",
pp.~310--317 in 
{\it Proc. IEEE Int. Conf. Computer Aided Design (ICCAD)}, 1992.

\bibitem{StKo86}
J.~A. Starzky and A. Konczykowska,
``Flowgraph analysis of large electronic networks",
{\it IEEE Trans. Circuits and Systems},
vol.~33, no.~3, pp.~302--315, March 1986.

\bibitem{ShiTan97}
C.-J.~Shi and X.-D.~Tan,
``Symbolic analysis of large analog circuits with
determinant decision diagrams,"
in {\it Proc. IEEE Int. Conf. Computer Aided Design (ICCAD)},
pp.~366-373, Nov. 1997.


\bibitem{ShiTan98_1}
C.-J.~Shi and X.-D.~Tan,
``Efficient derivation of exact s-expanded symbolic expressions
for behavioral modeling of analog circuits,"
in {\it Proc. IEEE Custom Integrated Circuits Conf. (CICC)},
pp.~463-466, May 1998.

\bibitem{ShiTan99_1}
X.-D.~Tan, C.-J.~Shi, D.~Lungeanu, J.-C.~F.~Lee, L.-P.~Yuan,
``Reliability-contrainted area 
optimization of VLSI power/ground networks via
sequence of linear programming,"
in {\it Proc. IEEE/ACM Design Automation Conf.}, 
New Orleans, LA, June 21-25, 1999.

\bibitem{TanTo97}
X.-D.~Tan, J.-R.~Tong, P.-S.~Tang and F.~Lombardi,
``An efficient multi-way algorithm for balanced partitioning
of VLSI circuits,"
{\it IEEE Int. Conf. Computer Design (ICCD)}, pp.~608-613, 1997.

\bibitem{TanShi98}
X.-D.~Tan and C.-J.~Shi,
``Hierarchical symbolic analysis of large analog circuits
with determinant decision diagrams,"
in {\it Proc. IEEE Int. Symp. Circuits and Systems}, pp.~318-321,
June 1998.

\bibitem{TanShi99_1}
X.-D.~Tan and C.-J.~Shi,
``Balanced multi-level, multi-way
partitioning of large analog circuits for hierarchical symbolic analysis,"
in {\it Proc. IEEE Asia and South Pacific Design Automation Conference
(ASP-DAC)}, Hong Kong, pp.~1-4, Jan. 1999.

\bibitem{TanShi99_2}
X.-D.~Tan and C.-J.~Shi, 
``Interpretable symbolic small-signal characterization of large analog
 circuits using determinant decision diagram," 
 in {\it Proc. IEEE Design, Automation and Test in Europe (DATE)}, 
 Munich, Germany, pp.~448-453, Mar. 1999. 

\bibitem{VlSi94}
J. Vlach and K. Singhal,
{\it Computer Methods for Circuit Analysis and
Design}, Van Nostrand Reinhold, New York, 1994.

\bibitem{WaGi92}
P.~Wambacq, G.~Gielen and W.~Sansen,
``A cancellation-free algorithm for the symbolic simulation
of large analog circuits",
pp.~1157--1160 in {\it Proc. IEEE Int. Symp. Circuits and Systems}, 1992.

\bibitem{WaGi96}
P.~Wambacq, G.~Gielen and W.~Sansen,
``A new reliable approximation
method for expanded symbolic network
functions",
pp.~584--587 in {\it Proc. IEEE Int. Symp. Circuits and Systems}, 1996.

\bibitem{YoSa95}
Z. You, E. S\'anchez-Sinencio,
and J.~P. de Gyvez,
``Analog system-level fault diagnosis based
on a symbolic method in the frequency domain",
{\it IEEE Trans. Instrumentation and Measurement},
vol.~44, no.~1, pp.~28--35, Jan. 1995.

\bibitem{YuSe96}
Q.~Yu and C.~Sechen,
``A unified approach to the approximate symbolic analysis of 
large analog integrated circuits",
{\it IEEE Trans. Circuits and Systems}, vol.~43, no.~8,
pp.~656--669, August 1996.

\end{thebibliography}
 
\end{document}
